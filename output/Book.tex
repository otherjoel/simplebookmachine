% A template for turning a series of dated posts/chapters (i.e. a blog)
% into a CreateSpace-ready 5.25" x 8" book.
% 
% Joel Dueck <joel@jdueck.net>
%
\documentclass[english,showtrims, 10pt]{memoir} 
\usepackage[breaklinks=true,unicode=true,hidelinks]{hyperref} 
\usepackage[T1]{fontenc}
\usepackage[sc]{mathpazo} 
\usepackage{xspace}
\usepackage[protrusion=true]{microtype}
\usepackage{xltxtra}

% Set the main and monospaced fonts
%
\setromanfont[Mapping=tex-text,Ligatures={Common, Rare, Discretionary},Numbers=OldStyle]{Adobe Caslon Pro}
\setmonofont[Mapping=tex-text,Scale=MatchLowercase]{Triplicate T4c}

% Make hyperlinks appear as footnotes
% Taken from https://groups.google.com/forum/#!topic/pandoc-discuss/O-N0H1eBnVU
\usepackage{url}
\renewcommand{\href}[2]{#2\footnote{\raggedright\url{#1}}}
\renewcommand\UrlFont{\rmfamily\itshape}

% Customize footnotes so that, within the footnote, the footnote number is
% the same size as the footnote text (per Bringhurst). Also remove the 
% horizontal rule above footnotes.
%
\usepackage[norule,splitrule,multiple,hang]{footmisc}
\makeatletter
\renewcommand\@makefntext[1]{\parindent 1em%
    \noindent
    \hb@xt@0em{\hss\normalfont\@thefnmark.} #1}
\def\splitfootnoterule{\kern-3\p@ \hrule width 1in \kern2.6\p@}
\makeatother
\renewcommand\footnotesize{\fontsize{8}{10} \selectfont}
\renewcommand{\thefootnote}{\arabic{footnote}}

% Here is where we configure our paper size and margins.
% The text block is sized and placed in a pleasing way
% according to the methods described here:
%  http://retinart.net/graphic-design/secret-law-of-page-harmony/
%
\setstocksize{8in}{5.25in}
\settrimmedsize{8in}{5.25in}{*}
\settypeblocksize{5.333in}{3.5in}{*}
\setlrmargins{0.583in}{*}{*}
\setulmargins{0.889in}{*}{*}
\setmarginnotes{0.156in}{0.75in}{\onelineskip}
\setheadfoot{\onelineskip}{2\onelineskip}
\setheaderspaces{*}{\onelineskip}{*}
\checkandfixthelayout

% The next few commands customize the chapter format a bit to suit 
% blog posts. Blog posts typically each have their own posted date,
% and we need to typeset that date somewhere. I have chosen to place
% the date where the chapter number would normally go in the 'dowding'
% chapter style (centered above the chapter title). Thus 'numbered' 
% chapters are now actually 'dated' chapters.
% 
% First we define a couple of custom commands:
%
\newcommand{\setChapterDescription}[1]{%
   \def\chapterDesc{#1}%
}
\def\chapterDesc{}

% Our chapter template will use the \ChapterDate command to include
% the date specified in the Markdown source for each chapter. (If 
% no date is specified, the template will use a "non-numbered" 
% chapter heading.)
%
\newcommand{\ChapterDate}[2]{
	\setChapterDescription{#2}
	\chapter{#1}
    %\addcontentsline{toc}{chapter}{#1}
	\setChapterDescription{}
}

% Here we customize the "dowding" chapter style to create our own
% (dowdingdate) which removes the word "chapter" and replaces the 
% chapter number with \chapterDesc. The changes are marked.
% 
\makeatletter
\makechapterstyle{dowdingdate}{%
  \setlength{\beforechapskip}{2\onelineskip}
  \setlength{\afterchapskip}{1.5\onelineskip \@plus .1\onelineskip 
                            \@minus 0.167\onelineskip}%
  \renewcommand*{\chapnamefont}{\normalfont}%
  \renewcommand*{\chapnumfont}{\chapnamefont}%
  
  % Remove the word "Chapter" before the date (where the chapter 
  % number would normally be)
  \renewcommand*{\printchaptername}{}%
  
  % Print the contents of \chapterDesc in place of the chapter number
  % except in appendices (where a simple [roman] numeral is printed)
  \renewcommand*{\printchapternum}{\centering\chapnumfont \ifanappendix \thechapter
                                   \else \chapterDesc\fi}
  % Original % \renewcommand*{\printchapternum}{\centering\chapnumfont \ifanappendix \thechapter
  %               \else \numtoName{\c@chapter}\fi}%
  \renewcommand*{\chaptitlefont}{\normalfont\itshape\huge\centering}%
  \renewcommand*{\printchapternonum}{%
    \vphantom{\printchaptername}\vskip\midchapskip}}
\makeatother


  \setsecnumdepth{chapter}
  %\setcounter{secnumdepth}{0}  % or -1

 

\makeatletter
\def\maketitle{%
  \null
  \thispagestyle{empty}%
  \vfill
  \begin{center}\leavevmode
    \normalfont
    {\LARGE\raggedleft \@author\par}%
    \hrulefill\par
    {\huge\raggedright \@title\par}%
    \vskip 1cm
%    {\Large \@date\par}%
  \end{center}%
  \vfill
  \null
  \cleardoublepage
  }
\makeatother

 \title{The Local Yarn}  
 \author{Joel Dueck}  
    \date{Feb 22, 2015} 

\begin{document} 
\chapterstyle{dowdingdate}

%\let\cleardoublepage\clearpage

 \maketitle  

\frontmatter

\null\vfill

\begin{flushleft}
 \textit{The Local Yarn} 


© \textsc{COPYRIGHT INFO}


ISBN--INFO

ISBN--13: 
\bigskip





\textsc{ALL RIGHTS RESERVED}




\end{flushleft}

 

\cleardoublepage
% Suppress "Contents" from referencing itself in the ToC
\begin{KeepFromToc}
  \tableofcontents
\end{KeepFromToc}

\mainmatter
\pagestyle{simple}

\ChapterDate{Imagination and Self-Doubt}{July 18, 2014}

{In a} 2005 episode of This American Life, titled ``A Little Bit of
Knowledge'',
\href{http://www.thisamericanlife.org/radio-archives/episode/293/a-little-bit-of-knowledge?act=3\#play}{we
hear the story of an electrician}, who, despite being fairly smart,
nonetheless deludes himself into thinking he has disproved Einstein's
theory of relativity. The fact that he's reasonably intelligent makes it
all the harder for him to see his own error, even when confronted by
actual experts.

\begin{quote}
\textbf{Bob Berenz:} All right, in this point I have to be completely
honest. I did write a paper early on, and I submitted it to a physics
site. And it was summarily rejected out of hand. But I did learn an
important lesson, that physicists and what's being done by them is very
complicated, very mathematically intensive. What I've got is none of
that, so it completely, almost in reverse, goes over their heads."
\end{quote}

Whenever his theory is challenged, Bob's response is to reject the
messenger as being narrow-minded or unintelligent. In fact, of course,
\emph{Bob} is the one whose mind is not quite up to the task he has set
for himself. But he can't allow himself even to suspect this. In this
episode's narrative, Bob goes from brushing off simple, obvious clues to
his own crackpottedness (the rejection of his paper) to dismissing
direct personal demonstrations of his ideas' incorrectnesses. Even after
a PhD in nuclear physics takes time to meet with him and explain what's
wrong with his theories, Bob comes away totally unfazed.

\begin{quote}
\textbf{Bob Berenz:} Well, this is not really fair, but I'm going to say
it anyway. It's like he {[}Dr.~Brant Watson{]} was talking the party
line. He didn't strike me as being all that bright. I know he has a
couple of patents, and he's this big professor, and it's probably not
fair for me to say that, but I'm not claiming to be this incredible
genius in this one area. It's very simple what I ran into. And I need
some help to get it put into a forum where people can understand it. But
it really isn't that difficult.
\end{quote}

I listened to all this with great interest, because I fear that I myself
might be just like Bob. In fact, I waver between thinking I am \emph{in
danger} of \emph{becoming like} Bob, and thinking I \emph{have been like
Bob for years} and am \emph{just now realizing it}. I have big,
long-nursed theories of my own about subjects I have no formal training
in, and I even write \& tweet on those subjects, with few to no
disclaimers.\footnote{Consider this article my life-long disclaimer.}

When an idea captivates your imagination, you can't really do anything
\emph{but} ponder it, discuss it, write about it, and live it out -- at
least, not until another idea captivates it even more strongly. And
there, I hope, is where Bob and I part company: because while he will
not even admit the possibility of a new idea displacing his old ones,
that possibility is deeply appealing to me. In fact I
\href{http://howellcreekradio.com/episodes/the-field-and-the-fortress}{genuinely
enjoy being proven wrong} much more than the feeling of being proven
apparently correct. In that moment there is a kind of clarity that is
real and rare. I may put up a fight and kick a lot of tires before I get
there, but really this is because I want that clarity: to \emph{know}
I'm wrong, and not to have to go on just \emph{suspecting} I'm wrong.

I regret that I don't have (or don't know that I have) the direct
knowledge or tools to construct with certainty my ideas on theology,
metaphysics or economics; I fear I will never have them. But being,
thankfully, aware of this shortcoming, I can do no other than to explore
these ideas with humility and the hope of further discovery; and I
invite my readers to humour me, and join with me in the same spirit.

\ChapterDate{Death, Decay and the Haunted Afterlife}{October 26, 2014}

{The common} Western idea of death is that once you die you are
transported immediately to a place of ultimate comfort (or torment),
with no intervals. Your consciousness continues seamlessly into the next
life, just no longer tied to your physical body.

And then `compiled' to {PDF}:

\begin{verbatim}
$ pandoc -s -o out.tex simplepost.pdf
$ xelatex out.tex out.pdf
\end{verbatim}

The result is a very basic {PDF} set in Computer Modern, suitable for
printing on loose sheets of 8.5″ × 11″ paper:

\begin{center}\rule{0.5\linewidth}{\linethickness}\end{center}

Suppose, though, that while consciousness does continue in some way
after death, it remains thoroughly joined to your physical remains. As
your body decays, so does your personality, your capacity to reason.
Your emotions, having been all along largely the product of your fluids
and nerves, transmute ever more into the mute horror which your remains
increasingly depict.

This consciousness-in-death was a running theme of Edgar Allen Poe. His
genius, in my view, was to paint this horror only at the edges --
premature burials,
\href{http://www.eapoe.org/works/tales/lssbthb.htm}{loss of breath}, the
unwillingness of the ``deceased'' to actually die -- so intimately that
we often mistake the edge for the abyss at the center. But in a kind of
artistic truth, the madness of Poe's narrators increases in proportions
as that abyss (the experience of actually being dead) is really
approached -- whenever we see a lady of Usher, for example, or hear the
beating of a murdered man's heart -- it is nearly always seen through
senses that are themselves being damaged.

\subsection{C.S. Lewis's view}\label{c.s.-lewiss-view}

C.S. Lewis, on the other hand, gets right inside this idea, describing
it in lucid detail in \emph{Perelandra}, when his villain Weston
temporarily recovers his senses and
\href{http://www.bestlibraryspot.net/fantasticfiction/Perelandra/14838.html}{tries
to articulate his experience of death}:

\begin{quote}
``\ldots{}All the good things are now -- \textbf{a thin little rind of
what we call life}, put on for show, and then -- the real universe for
ever and ever. To thicken the rind by one centimetre -- to live one
week, one day, one half hour longer -- that's the only thing that
matters. \ldots{} {[}Humanity{]} knows -- Homer knew -- that \textbf{all
the dead have sunk down into the inner darkness: under the rind.} All
witless, all twittering, gibbering, decaying\ldots{} Then there's
Spiritualism\ldots{}I used to think it all nonsense. But it isn't. It's
all true. You've noticed that all pleasant accounts of the dead are
traditional or philosophical? What actual experiment discovers is quite
different. Ectoplasm - slimy films coming out of a medium's belly and
making great, chaotic, tumbledown faces. Automatic writing producing
reams of rubbish.''
\end{quote}

I wonder whether this idea of death is common in other cultures, or
times of history. It seems to me something very Other -- definitely not
Christian, but not exactly Pagan either.

The song `Bottom of the River' by Tom Fun Orchestra seems an appropriate
example as well. Interpreted by
\href{http://www.animationblog.org/2009/10/alasdair-brotherston-jock-mooney-bottom.html}{some}
as concerned mainly with environmental vandalism, it seems to me more
straightforward to read it as the experience of those ``beneath the
rind'': dead yet conscious, still chained to their remains, enduring a
never-ending passage of time where light is distant and the idle
time-marking movements of the living all too close. 

 

\end{document}